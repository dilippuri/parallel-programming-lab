\documentclass{sem5}
\institutename{Indian Institute of Information Technology, Vadodara}
\author{Hemant Kumar}
\idt{201352026}
%\team{teamname}
\collab{\textbf{Collaborator} - Dilip Puri(201351014)}

\coursename{Parallel Programming}
\ccode{\begin{small}CS403\end{small}}
\profname{Prof. Reshmi Mitra}

\type{Lab}
\typeid{06}
\submissiondate{\today}%dd/mm/yyyy
\deadline{Nov 26, 11:59 PM}%dd/mm/yyyy @hh:mm pm/am
\problemset{Compiler Optimization}

\begin{document}
\section*{Introduction}
In computing, an optimizing compiler is a compiler that tries to minimize or maximize some attributes of an executable computer program. The most common requirement is to minimize the time taken to execute a program; a less common one is to minimize the amount of memory occupied.

\section*{Optimization Table}



\section*{Methods}
These options control various sorts of optimizations:\\
-O\\
-O1\\
Optimize. Optimizing compilation takes somewhat more time, and a lot more memory for a large function.\\
Without -O, the compiler's goal is to reduce the cost of compilation and to make debugging produce the expected results. Statements are independent: if you stop the program with a breakpoint between statements, you can then assign a new value to any variable or change the program counter to any other statement in the function and get exactly the results you would expect from the source code.\\

With -O, the compiler tries to reduce code size and execution time, without performing any optimizations that take a great deal of compilation time. \\

-O2\\
Optimize even more. GCC performs nearly all supported optimizations that do not involve a space-speed tradeoff. The compiler does not perform loop unrolling or function inlining when you specify -O2. As compared to -O, this option increases both compilation time and the performance of the generated code.\\
-O2 turns on all optional optimizations except for loop unrolling, function inlining, and register renaming. It also turns on the -fforce-mem option on all machines and frame pointer elimination on machines where doing so does not interfere with debugging.\\

Please note the warning under -fgcse about invoking -O2 on programs that use computed gotos. \\

-O3\\
Optimize yet more. -O3 turns on all optimizations specified by -O2 and also turns on the -finline-functions and -frename-registers options. \\
-O0\\
Do not optimize. \\
-Os\\
Optimize for size. -Os enables all -O2 optimizations that do not typically increase code size. It also performs further optimizations designed to reduce code size.\\
If you use multiple -O options, with or without level numbers, the last such option is the one that is effective.\\
\section*{Examples}
\begin{lstlisting}
Loop unrolling
Example:
	// old loop 
	for(int i=0; i<3; i++) {
			colormap[n+i] = i;
	}
	// unrolled version
	int i = 0;
	colormap[n+i] = i;
	i++;
	colormap[n+i] = i;
	i++;
	colormap[n+i] = i;
\end{lstlisting}	
	 
\end{document}